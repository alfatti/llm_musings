\documentclass{article}
\usepackage{amsmath,amsfonts}
\usepackage[margin=1in]{geometry}

\begin{document}

\section*{Methodology: Causal Graph Convolution (``CANAL'') for Dynamic Bond Recommendation}

\subsection*{Problem Setting}

Let $(u,i,t)$ denote an interaction where client\footnote{In this context a ``user’’ is an institutional client and an ``item’’ is a tradable corporate bond.} 
$u\in\mathcal U$ shows interest in bond $i\in\mathcal I$ on business-day $t\in\mathbb Z_{\ge 0}$.  
Daily interactions form a sequence of bipartite graphs
$
\{\,G_t=(V_t,E_t)\}_{t=1}^{T},\quad 
V_t=\mathcal U_t\cup\mathcal I_t,\;
E_t=\{(u,i)\mid(u,i,t)\text{ observed}\}.
$
The objective is to rank \emph{currently live} bonds for each client on day~$t$ using only information strictly prior to~$t$.:contentReference[oaicite:0]{index=0}

\subsection*{Rolling Temporal Window}

To avoid data leakage and to keep recommendations responsive, CANAL constructs for every
prediction day $t$ a \emph{causal snapshot}
\[
G_{t,w}=G\bigl[\,\{(u,i,t')\mid t-w\le t' < t\}\bigr],
\]
where $w$ is a sliding-window length (typically $1\!-\!5$ trading days in credit markets).\footnote{A smaller $w$ emphasises immediacy but may sacrifice long-term signals; a larger $w$ risks diluting recent trends.}  
Only $G_{t,w}$ is exposed to the model when generating rankings for day~$t$.:contentReference[oaicite:1]{index=1}

\subsection*{Message Passing}

CANAL re-uses the LightGCN backbone but replaces the usual static neighbourhood aggregate with a \emph{time-aware causal aggregate}.  
For $k=1,\dots,L$ layers and each node $v$ (client or bond) we compute
\[
h^{(k)}_v(t)=
    \sum_{(v',\Delta t)\in\mathcal N_{t,w}(v)}
    c^{(\Delta t)}_{vv'}\;h^{(k-1)}_{v'}(t),
\quad 
c^{(\Delta t)}_{vv'}=\frac{1}{\sqrt{|\mathcal N_{t,w}(v)|\,|\mathcal N_{t,w}(v')|}},
\]
where $\mathcal N_{t,w}(v)$ lists edges incident to $v$ inside $G_{t,w}$ and
$\Delta t=t-t'$ records the age of each interaction.  
The key differences from static LightGCN are:

\begin{itemize}
    \item \textbf{Time index $t$:} embeddings $e_v(t)$ evolve daily, allowing the same client or bond to carry different representations on different days.
    \item \textbf{Causal neighbourhood:} future interactions $(t'>t)$ are never accessed, eliminating look-ahead bias.
    \item \textbf{Rolling window restriction:} only edges newer than $w$ days are considered, so stale preferences decay naturally.
\end{itemize}:contentReference[oaicite:2]{index=2}

The final embedding for node $v$ on day $t$ is the (LightGCN) layer-wise weighted sum
$
e_v(t)=\sum_{k=0}^{L}\alpha_k\,h^{(k)}_v(t),\;
\alpha_k=\tfrac1{k+1}.
$
The predicted relevance score is the dot product
$
\hat a_{u i}(t)=e_u(t)^\top e_i(t).
$

\subsection*{Training Objective}

Daily quadruplets $(t,u,i,j)$ are sampled with
\(
(i\! \in\!  E_t,\;j\notin E_t)
\)
and optimised using Bayesian Personalised Ranking (BPR):
\[
\mathcal L_{\mathrm{BPR}}
    =-\!\!\!\!\sum_{(t,u,i,j)}\!
      \log\sigma\!\bigl(\hat a_{u i}(t)-\hat a_{u j}(t)\bigr).
\]
Dynamic negative sampling at a $10{:}1$ ratio and Adam optimisation are applied; early stopping is gauged on temporal validation slices.:contentReference[oaicite:3]{index=3}

\subsection*{Why It Matters to Bond Desks}

\begin{itemize}
    \item \textbf{No look-ahead}: CANAL mimics the trader’s real information set—recommendations never exploit trades that have not yet happened.
    \item \textbf{Drift control}: the rolling window guards against regime shifts (e.g.\ earnings releases, rating actions) that quickly render older trades irrelevant.
    \item \textbf{Lightweight}: the model keeps LightGCN’s linear, feature-free layers—training time scales linearly in edges, making daily refresh feasible for thousands of clients and tens of thousands of bonds.
    \item \textbf{Interpretable signals}: dot-product scores can be decomposed into path contributions, giving hints about which recent peer trades support a recommendation—useful for sales conversation.
\end{itemize}

\end{document}
